\section{Motivation}

Many million users are using browsers in their daily life for various kinds of activities. Attackers are trying to make use of the browsers vulnerabilities, often using severe 0-day exploits to gain advanced acccess - sometimes to the system itself.

Code injection is an old unsolved problem on Windows computers. Many different injection techniques exist and client side protection proves to be very difficult, as defense mechanism have to directly tackle the injection or injecting application. Anti-cheat applications try to do that but often are limitted to very specific games and detect only a very minor part of illegal code injections. This is due to the shear unlimited number of ways code can be injected. In general, the injection techniques can be categorized into two major groups, dll injection and the even more powerful direct code injection. DLL injection typically uses one of the ways to get into the target process. SetWindowsHookEx registers a system wide hook, that gets called whenever a certain event type has been fired and the calling SetWindowsHookEx function has registered for that. The Dll can then be loaded from inside the callback function and is - due to this functionality - injected into the remote process. A more advanced version uses CreateRemoteThread from the windows API, to start a thread in a remote process. Inside the remote thread a LoadLibrary call is made and the injection succeeds with the loading of the dll. A remote process can detect and prevent these techniques with appropriate defense mechanism, but attackers might simply improve their injection technique by hiding information about the dll being loaded. The second group of injections uses direct code injection into the remote process. 