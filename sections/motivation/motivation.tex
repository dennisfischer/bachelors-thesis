\section{Motivation}

Many million users are using browsers daily to access internet pages and attackers are trying to make use of it, by using several different ways of attacking the browser itself. Often existing browser bugs (0-day exploits) or bugs in one of it's commonly used plugins like the Adobe Flash Player, to gain control of at least browser and often system functionalities. Another also very common way is the installation of unwanted toolbars without permissions by the controlling user. A possible way of doing so is dll-injection, which has been more commonly used in (online) game cheating, but can be used with arbitary applications to do malicious activities like openening and replacing ads or pishing techniques. This thesis will try to analyze the threads of Google Chrome's browser and do provenance of the existing threads, to detect unwanted dll-injections.