\subsection{DLL Blacklist}
Chrome ELF contains a DLL blacklist feature, which automatically prevents unwanted DLLs from being run inside Chrome. The blacklist is split into two parts, the first contained inside the code and a second list loaded from the local computers registry. The existing detection algorithms checks for an existing entry in the blacklist by comparing the file- and imagename of the loaded DLL file. If a blacklisted DLL is found, [........] unloads the DLL from Chrome and ensures that it is no longer contained inside the process' virtual memory. However, this algorithms adds no security! A comparison between file-/imagenames is very weak, as an attacker might simply change the name of the DLL file. Additionally, every new build of the blacklisted module will not be detected if the name is changed. The whole purpose of the existing blacklist feature is to increase stability of Google Chrome and prevent instable extensions and addons from running, to ensure an as optimal as possible user experience inside the browser.