\subsubsection{DLL Blacklist}
Chrome ELF contains a DLL blacklist feature, which prevents unwanted DLLs from being loaded into Chrome. The blacklist is split into two parts, the first contained inside the code and a second list loaded from the local computers registry. The existing detection algorithms checks for an existing entry in the blacklist by comparing the file- and imagename of the loaded DLL file. If a blacklisted DLL is found, it is unloaded from Chrome. A comparison between file-/imagenames is  weak, because an attacker can change the name of the DLL file. Additionally, every new build of the blacklisted DLL will not be detected if the name is changed. The whole purpose of the existing blacklist feature is to increase stability of Google Chrome and prevent unstable extensions and addons from running, to ensure an as optimal as possible user experience.