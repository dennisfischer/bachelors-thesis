\subsubsection{Software Based Protection against Changeware}
\cite{changeware} shows a "software-based protection against changeware, i.e. malware with no root/administrator privileges, which performs surreptitious persistent changes to software assets which are not protected by the OS code signature verification mechanism." The main focus of this work is ensuring the integrity of an application X by using white-box-cryptography \cite{wbcrypto}, software diversity \cite{Forrest} and run-time process memory invariant checking. The proposed solution does not rely on specific hardware like, e.g. \emph{Trusted Platform Modules}, and can be applied by the vendor of the software. An effective way is presented to prevent automatic attacks on different instances of an application X. This concepts were applied to \emph{Chromium} to prevent changeware from changing the browsers default settings, e.g. the search engine, which will gain the attacker huge revenues.\\
However, this defense can only ensure integrity of applications and not their confidentiality or availability. Attacks with \syscall{CreateRemoteThread} or \syscall{WriteProcessMemory} are still possible as they are not blocked. An assumption about the security of the designed protection is made. As long as the attacker is unable to retrieve the private key, which is stored inside a thread's local storage, s/he will not be able to break integrity of an application and due to the overall design, key extraction are attacks are hardly possible. The thread's local storage is hereby at a random location for each thread, forcing the attacker to search the whole memory space, which is not feasible. However, a skillful attacker might be able to hijack a thread with \syscall{CreateRemoteThread} and \syscall{WriteProcessMemory} in such a way, that the thread's local storage is accessed and the proposed integrity protection is no longer existent.