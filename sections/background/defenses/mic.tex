\subsubsection{Mandatory Integrity Control}
\begin{figure}[!htbp]
\centering
\includegraphics[scale=0.5]{sections/background/defenses/mic.png}
\caption{The hierarchy of integrity levels with decreasing permissions from top to bottom}
\label{fig:mic}
\end{figure}
\gls{MIC} is based on the Bell-LaPadula \cite{eckert2014sicherheit} model and multi level security. On Windows systems, there are four different integrity levels with decreasing privileges: System, High, Medium and Low. Figure \ref{fig:mic} shows an example of the integrity hierarchy, sorted form top to bottom by decreasing permissions. By definition it is not possible to access an object of higher integrity level than the accessing object. Therefore, if process memory tampering is used, both processes need to have at least the same integrity level. The fourth integrity level \syscall{System} cannot be reached from the logged in user. Processes can now be started with one of the remaining integrity levels. By default, all processes receive \syscall{Medium} integrity level. \syscall{Low} needs to be explicitly assigned and \syscall{High} is used only if the process is started with elevated permissions. Thus, as most processes run with \syscall{Medium} integrity level, \gls{MIC} can not be used to preventing memory tampering.