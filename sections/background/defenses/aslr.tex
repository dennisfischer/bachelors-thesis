\subsubsection{Adress Space Layout Randomization}
The first of the built in windows defense mechanism is called Address Space Layout Randomization. It's main purpose is to make memory allocations at a random address in the virtual address space. Figure \ref{fig:aslr} shows three independent starts of the operating system and the locations of the loaded DLL files which is randomized. An attacker that tries injecting code into the application will no longer know which address has to be used. Searching inside the virtual address space is thus not feasible, as it is very large with four gigabyte on a 32 bit and eight terabyte on a 64 bit windows system of virtual addresses. Therefore the attacker might start duplicating his code several times to fill hundreds of addresses and by chance getting his code executed. As the main purpose of ASLR is to prevent buffer overflows, it is fulfilling its job by crashing the application due to an illegal memory access. However, for all the given attacks, ASLR is useless! All methods involving \syscall{WriteProcessMemory} or DLL-Injection can use the Windows API function \syscall{GetProcAddress} to receive the address of an exported function. As the ecosystem is very DLL dependent and procedures are exported from within DLLs, the attacker is granted an easy way to circumvent ASLR.
\begin{figure}[h]
\centering
\includegraphics[width=\textwidth,height=\textheight,keepaspectratio]{sections/background/defenses/aslr.jpg}
\caption{ASLR resulting in different DLL memory locations after each boot}
\label{fig:aslr}
\end{figure}