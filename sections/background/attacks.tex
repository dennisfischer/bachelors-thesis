\subsection{Attacks}
\label{sec:attacks}
This subsection will deal with existing attacks that can be used to gain control over a process, allowing to change values and code in memory as well as controlling which parts are executed. The attacks can be split into two groups, external and internal modifications. External modifications describe the attacks of Figure~\ref{fig:attacks_external}. The difference of external and internal modifications is the starting point of the exploit. External modification attacks start their exploit from outside the targets process. This can be another process which uses \gls{API} functions or predefined \emph{Windows} structures that allow \gls{WPM}, \gls{DLL} Injection and Buffer Overflow attacks. In contrast, internal modification attacks start from inside the targets process with typical functions being system calls.
\begin{figure}[h]
\centering
\includegraphics[width=\textwidth,keepaspectratio]{sections/adtrees/ExternalModificationsWithoutDefenses.png}
\caption{This attack tree shows all possible external attacks to the original software}
\label{fig:attacks_external}
\end{figure}
\subsubsection{\syscall{WriteProcessMemory} external modifications}
\gls{WPM} is just like the previously discussed \syscall{SetWindowsHookEx} a function available through the Windows \gls{API}. This function allows to write into the virtual memory of a target process and allows modifying existing data or code. There are several ways to make use of this function. \gls{WPM} is used for \gls{DLL} injection with the \syscall{CreateRemoteThread}, code modification and function detouring. 
\paragraph{\syscall{CreateRemoteThread} \gls{DLL} Injection}
\glspl{DLL} can be loaded into a target process by loading their path into the target memory with \gls{WPM} and creating a new thread via \syscall{CreateRemoteThread} that calls \syscall{LoadLibrary}. This kind of attack can be used on any process running under the same integrity level and therefore is widely used in game cheating.

\paragraph{Step 1} 
At first, a handle to the target process is requested via \syscall{OpenProcess} with \syscall{PROCESS\_VM\_WRITE} and \syscall{PROCESS\_VM\_OPERATION} access rights. These access right are required to execute the \gls{WPM} function. If the permissions are missing, \gls{WPM} is not able to modify the virtual memory of the target process. Virtual memory protection flags do not have to be changed manually, as this is already happening inside \gls{WPM}. 

\paragraph{Step 2} 
A large enough amount of memory is allocated inside the target process with \syscall{VirtualAllocEx}, to hold the full path of the \gls{DLL}. 

\paragraph{Step 3}
Next, \gls{WPM} is used to transfer the \gls{DLL} path into the target memory space.

\paragraph{Step 4}
The injection can be completed by calling \syscall{CreateRemoteThread}, which loads the \gls{DLL} with the \syscall{LoadLibrary} function. The allocated memory segment is used as a parameter for the \syscall{LoadLibrary} function call. 

\paragraph{Step 5}
With the loaded \gls{DLL}, arbitrary code can get executed by the \gls{DLL} inside the attacked process, either via using \syscall{CreateRemoteThread} again or via the \gls{DLL}'s entry point.


An example of a basic \gls{DLL} injection using \gls{WPM} and \syscall{CreateRemoteThread} can be found in Appendix~\ref{appendix:writeprocessmemory}. \emph{Chrome} shows no existing defense mechanisms against direct memory modification.
\paragraph{Code modification with \syscall{WriteProcessMemory}}
A second way to make use of \gls{WPM} is modifying the targets code inside memory, to execute different instructions than intended. This technique is also commonly known as function hooking or function detouring \cite{codeproject_hooking}.
\paragraph{Function Detouring}
\begin{figure}[h]
	\centering
	\includegraphics[keepaspectratio,width=.6\textwidth]{sections/background/attacks/fig_detours.png}
	\caption{This figure shows the change in control flow of a detoured function \cite{detours}}
	\label{fig:detours}
\end{figure}
Function detouring has been greatly simplified by the \emph{Detours} \cite{msdetours} library of \emph{Microsoft}, but can also be achieved by memory modification with \gls{WPM} or assembly code. Figure~\ref{fig:detours} shows the difference between a function before and after detouring. The function call at the top shows an invocation without interception. The source function calls the target function without any indirection and after the code of the target function has been executed, returns to the calling source function. Detouring makes use of this structure by placing a detour and a trampoline function in between these calls. The source function will now use an indirect call to the target function, by first calling the detour part, which gives space to execute arbitrary code. To do that, a \syscall{jmp} instruction is placed at the beginning of the source function, and the original instructions are saved and copied to the trampoline function. After that, the detour function continues with the trampoline function, which executes the copied instructions and ensures that the target function works as if there was no detour placed. Finally, the whole function stack will return, this time skipping the trampoline function, as it was just used to hold the copied instructions.
\subsubsection{\emph{Registry} Based Injection}
\emph{Registry} based injection uses a special \emph{Registry} key\footnote{The \emph{Registry} key can be found at \syscall{HKEY\_LOCAL\_MACHINE\textbackslash Software\textbackslash Microsoft\textbackslash\allowbreak Windows NT\textbackslash CurrentVersion\textbackslash Windows}} that can be used to inject \gls{DLL} files into almost any process. For that reason the shown attack tree in Figure~\ref{fig:attacks_external} lists this technique under the group of \gls{DLL} injections in node [1.2.1]. \glspl{DLL} can be added by writing the full path to the file into this \emph{Registry} key, with multiple paths separated by semicolons. To make this injection technique work, two conditions have to apply:
\begin{enumerate}
\item The \gls{DLL} file is having a security descriptor that allows execution. Otherwise the \gls{DLL} will not get loaded into the target process.
\item The \gls{MSDN} \cite{msdn_appinitdlls} lists the usage of \syscall{User32\allowbreak.dll} in the target process as a requirement or otherwise the \glspl{DLL} listed in the \emph{Registry} will not get loaded.
\end{enumerate}
However, these two conditions are easy to fulfill. The first condition is by default true, because the default security descriptor of \gls{DLL} files allow execution. Active interaction by other programs or the user is required to change the security descriptor. The second condition is most of the time fulfilled. \syscall{User32.dll} is used in almost every process and in general it can be assumed that the \gls{MSDN} requirement is fulfilled. To make this attack work, a second \emph{Registry} key, \syscall{LoadAppInit\_DLLs}, has to be changed to value 1, in order to activate this \emph{Registry} based injection. By default this value is set to 0 and cannot be changed without admin privileges.
Defending against this kind of attack is comparably easy, by checking \syscall{AppInit\_DLLs} value and enumerating all loaded modules (\glspl{DLL}). If a match is found and considered to be unwanted, it can be unloaded or as a safety measurement the application is terminated. As for \emph{Google Chrome}, there is no validation currently in place and \emph{Registry} based injection can be used to load arbitrary \gls{DLL} files.
\subsubsection{\syscall{SetWindowsHookEx} injection}
A more reliable way than registry based injection is using the \syscall{SetWindowsHookEx} function of the Windows API. It requires no special privileges to be executed and can be used to only hook into a specific application or being system wide. A hook in this context is a callback function that gets executed whenever a certain event is occurring. In terms of \syscall{SetWindowsHookEx}, there are according to the MSDN \cite{msdn_setwindowshookex} 15 different possible types of events that can be registered for, of which some can only be system wide. The hook procedure that is required as parameter for \syscall{SetWindowsHookEx} has to be located in a DLL file. The DLL to be injected is loaded inside the current process to locate the address of the callback function. An example code is shown in Appendix \ref{appendix:setwindowshookex}. As this attack is also a based on DLL injections the attack tree of Figure \ref{fig:attacks_external} list this attack under external modifications. 

The DLL will not get loaded into the target process until the event is triggered and the registered hook is called for the first time. The callback procedure gives different opportunities to make use of this just created situation. An additional thread can be started from the callback procedure or the called driver entry function to make the injection independent of hook callback function. As this module and especially any created thread are running the context of the remote process, the DLL code has now full access on the process virtual memory. Internal memory modifications can now be performed which will get explained in detail in Section \ref{sec:internal_modifications}. As well as for registry injection, chrome doesn't prevent \syscall{SetWindowsHookEx} DLL injections.
\subsubsection{DLL Replacement}
The last missing \gls{DLL} injection technique of the given attack tree in Figure~\ref{fig:attacks_external} is \gls{DLL} Replacement (node [1.2.3]). The idea is to replace an existing \gls{DLL} file with a patched or completely different \gls{DLL}, which is possible due to \emph{Windows} internal concepts. Every \gls{DLL} exports functions that can be called from a program. The attacker creates a new \gls{DLL} file, that exports the same functions and internally redirects all calls to the original \gls{DLL} file. Therefore the functionality of the application stays the same and the attack is harder to detect. It is now possible to use one of the exported functions to execute other, new code that performs malicious activities.
\subsubsection{Buffer overflows}
Buffer overflows are among the most important security problems of modern applications \cite{pethia}, as they are easy to create \cite{bufferoverflows_easy}. Buffer overflows allow the attacker to execute arbitrary code inside the attacked application and as such are running under the same privilege level. Regardless the fact if code execution is possible, sensitive information might get revealed. One of the most severe examples of the past was the so called \emph{Heartbleed bug} \cite{durumeri}, which affected millions of servers worldwide, as it was present in the much used \emph{OpenSSL}\footnote{\url{https://www.openssl.org/}} library. Even though the attacker could not execute code, he was able to get sensitive information about the SSL certificates private key. Encrypted data was no longer secure of external modifications or Man-in-the-Middle attacks. Buffer overflow attacks represent the node [1.3] of Figure \ref{fig:attacks_external} and are also a external modification. In contrast to the other previously shown attacks there are several countermeasures existing to mitigate the resulting exploit, which will be discussed in chapter \ref{sec:defenses}.
\subsubsection{Internal modifications}
\label{sec:internal_modifications}
\begin{figure}[!htbp]
\centering
\includegraphics[width=\textwidth, keepaspectratio]{sections/adtrees/InternalModificationsWithoutDefenses.png}
\caption{This attack tree shows possible attacks that are grouped under internal modification}
\label{fig:attacks_internal}
\end{figure}
A group of attack is internal modifications which is in contrast to external modifications occurring from inside the process virtual memory. As the attacker is already inside the virtual memory, modification is easier and less restricted than the previously shown external modifications. The attacker can make use of existing functions like \syscall{memcpy}\footnote{\url{http://www.cplusplus.com/reference/cstring/memcpy/}} or \syscall{memset}\footnote{\url{http://www.cplusplus.com/reference/cstring/memset/}}, to modify the values inside memory, without having to use the indirection via \syscall{WriteProcessMemory}. Besides the present \syscall{mem*} functions, the attacker can also make use of assembly code. Figure \ref{fig:attacks_internal} shows this type of attack in node [2.1]. The assembly node [2.2] shows an alternative way to achieve the same result as [2.1]. With assembly, instruction are executed and the indirection of using function calls is removed.  

\medskip

The shown attacks are combined in Figure \ref{fig:attacks}, containing now both parts of attacks, external and internal modifications.
\begin{figure}[!p]
	\centering
	\includegraphics[angle=90,height=\textheight,keepaspectratio]{sections/adtrees/ProcessVirtualMemoryWithoutDefenses.png}
	\caption{An attack tree combining all of the shown attacks}
	\label{fig:attacks}
\end{figure}
