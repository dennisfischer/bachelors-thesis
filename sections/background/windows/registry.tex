\subsubsection{Registry}
Storing important configuration data on Windows is done through the so called Registry. The registry stores information on a key-value basis and all configuration of Windows and most applications can be found in it. The registry has one major advantage in performance compared to the classic approach to store configuration information in \syscall{INI} files. Whenever a key gets changed, the \syscall{INI} file gets loaded into memory, modified and written back to the disk. This is not needed for changing values in the registry as the registry is already loaded during system start into memory and made changes are written with a delayed to disk. Additionally, the registry improves handling of multi user setups, which is hardly possible with a single \syscall{INI} file. To give the registry an initial layout, six major groups are defined by \emph{Windows 7}.
\begin{itemize}
\label{sec:registrykeys}
\item HKEY\_LOCAL\_MACHINE
\item HKEY\_CURRENT\_CONFIG 
\item HKEY\_CLASSES\_ROOT 
\item HKEY\_CURRENT\_USER 
\item HKEY\_USERS
\item HKEY\_PERFORMANCE\_DATA (invisble in \syscall{regedit.exe})
\end{itemize}
To access this registry by a user, the \syscall{regedit.exe} \cite{msdn_regedit} program can be used. Of the given six groups, only five can actually be seen, as the \syscall{HKEY\_PERFORMANCE\_DATA} is invisible in the registry editor. Some keys may require additional permissions like Administrator privileges to be accessible and changeable. This is a result of the previously explained security descriptor in Section \ref{sec:sd}.