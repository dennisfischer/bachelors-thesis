\subsubsection{Buffer overflows}
Buffer overflows are among the most important security problems of modern applications \cite{pethia}, as they are hard to detect, and are introduced by simple programming errors that were not previously found in testing and quality assurance steps. Even though they are hard to find, once used they allow the attacker to execute arbitrary code. Regardless the fact if code execution is possible, sensitive information might get revealed. One of the most severe examples of the past was the so called \emph{Heartbleed bug} \cite{durumeri}, which affected millions of servers worldwide, as it was present in the much used \emph{OpenSSL}\footnote{\url{https://www.openssl.org/}} library. Even though the attacker could not execute code, he was able to get sensitive information about the SSL certificates private key and thus break any security put in place by the SSL protocol. This attack is the last one of Figure \ref{fig:attacks_external} and also a external modification. In contrast to the other previously shown attacks there are several countermeasures existing to mitigate the resulting exploit, which will be discussed in chapter \ref{sec:defenses}.